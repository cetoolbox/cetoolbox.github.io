\documentclass{article}
\usepackage[utf8]{inputenc}
\usepackage[numbers]{natbib}
\usepackage{url}
\bibliographystyle{unsrtnat}

\title{CEToolbox Equations}
\author{Y. François and J. Pansanel}
\date{v1.4.5}
\usepackage[a4paper, margin=1in]{geometry}
\usepackage[hidelinks]{hyperref}
\usepackage{siunitx}

\begin{document}

\maketitle


\section{CEToolbox}

The CEToolbox application\citep{cetoolbox} is a calculator for capillary electrophoresis\citep{capillaryelectrophoresis}. It permits to calculate several parameters of a capillary electrophoresis analysis, like the hydrodynamic injection, the volume of the capillary, the quantity of injected analyte or the electroosmotic mobility.

This document presents the theoretical equations used by CEToolbox. For questions or remarks, please contact us through the CEToolbox project website.


\section{Equations}

\subsection{Hydrodynamic injection}

The volume of injected sample (\si{\nano\liter}) is calculated following the Poiseuille's law\citep{poiseuille}:

\[ V_{inj} = \frac{{\pi}d_c^4t{\Delta}P}{128{\eta}L} \times 10^{12} \]

where \(d_c\) is the internal diameter of the capillary (\si{\meter}), \(t\) the time to window (\si{\s}), \({\Delta}P\) the applied pressure (\si{\pascal}), \(\eta\) the viscosity (\si{{\centi}P}) and \(L\) the total length of the capillary (\si{\meter}).


\subsection{Capillary volume}

The volume of the capillary (\si{\nano\liter}) is computed accordingly to:

\[ V_{tot} = \frac{{\pi}d_c^2L}{4} \times 10^{12} \]

where \(d_c\) is the internal diameter of the capillary (\si{\meter}) and \(L\) the total length of the capillary (\si{\meter}).


\subsection{Capillary volume to window}

The capillary volume to detection window (\si{\nano\liter}) is determined with:

\[ V_{d} = \frac{{\pi}d_c^2l}{4} \times 10^{12} \]

where \(d_c\) is the internal diameter of the capillary (\si{\meter}) and \(l\) the length to detection window (\si{\meter}).


\subsection{Injection plug length}

The injection plug length (\si{\milli\meter}) is calculated with the following formula:

\[ l_{inj} = \frac{4V_{inj}}{{\pi}d_c^2} \times 10^{-9} \]

where \(V_{inj}\) is the volume of sample injected (\si{\nano\liter}) and \(d_c\) is the internal diameter of the capillary (\si{\meter}).


\subsection{Plug}

The percentage of plug is the ratio between the volume of injected sample and the capillary volume:

\[ \%_{tot} = \frac{V_{inj}}{V_{tot}} \times 100 \]

where \(V_{inj}\) is the volume of sample injected (\si{\nano\liter}) and \(V_{tot}\) the volume of the capillary (\si{\nano\liter}).


\subsection{Plug relative}

The percentage of relative plug is the ratio between the volume of injected sample and capillary volume to window:

\[ \%_{d} = \frac{V_{inj}}{V_d} \times 100 \]

where \(V_{inj}\) is the volume of sample injected (\si{\nano\liter}) and \(V_d\) the capillary volume to window (\si{\nano\liter}).


\subsection{Time to replace one volume}

The time (\si{\second}) to replace one volume is computed with the following equation, obtained by combining the capillary volume and the hydrodynamic injection equations:

\[ t = \frac{32{\eta}L^2}{d_c^2{\Delta}P} \times 10^{-3} \]

where  \(\eta\) is the viscosity (\si{{\centi}P}), \(L\) the total length of the capillary (\si{\meter}), \(d_c\) the internal diameter of the capillary (\si{\meter}) and \({\Delta}P\) the applied pressure (\si{\pascal}).


\subsection{Injection flow rate}

The injection flow rate (\si{\nano\liter/\min}) is calculated with:

\[ Q_{inj} = \frac{V_{tot}}{t} \]

where \(V_{tot}\) is the volume of the capillary (\si{\nano\liter}) and \(t\) the time (\si{\min}).


\subsection{Electrical field}

The electrical field (\si{\volt\per\centi\meter}) is determined by:

\[ E = \frac{U}{L} \times 10^{-2} \]

where  \(U\) the voltage (\si{\volt}) and \(L\) the total length of the capillary (\si{\meter}).


\subsection{Analyte injected}

The quantity of injected analyte (\si{\nano\gram}) is computed with:

\[ m = CV_{inj} \]

where \(C\) is the concentration (\si{\gram/\liter}) and \(V_{inj}\) the volume of injected sample (\si{\nano\liter}).


\subsection{Electroosmotic mobility}

The electroosmotic mobility (\si{\meter\squared\per\volt\per\second}) is the ratio between the electrophoretic velocity and the electrical field:

\[ \mu_{eof} = \frac{lL}{t_{eof}U} \]

where \(l\) is the length to detection window (\si{\meter}), \(L\) the full length of the capillary (\si{\meter}), \(t_{eof}\) the electroosmosis time (\si{\second}) and \(U\) the voltage (\si{\volt}).


\subsection{Separation flow rate}

The separation flow rate (\si{\nano\liter\per\minute}) is determined with the formula:

\[ Q_{sep} = \frac{{\pi}d_c^2\mu_{eof}E}{4} \times 60 \times 10^{14} \]

where \(d_c\) is the inner diameter (\si{\meter}), \(E\) the electrical field (\si{\volt\per\centi\meter}) and \(\mu_{eof}\) the electroosmotic mobility (\si{\meter\squared\per\volt\per\second}).


\subsection{Viscosity}

The viscosity (\si{{\centi}P}) is computed with:

\[ \eta = \frac{d_c^2t_m{\Delta}P}{32lL} \times 10^3 \]

where \(d_c\) is the inner diameter (\si{\meter}), \(t_m\) the migration time (\si{\second}), \({\Delta}P\) the applied pressure (\si{\pascal}), \(l\) the length to detection window (\si{\meter}) and \(L\) the full length of the capillary (\si{\meter}), 


\subsection{Conductivity}

The conductivity (\si{\milli\siemens\per\meter}) is calculated with the formula\citep{conductivity}:

\[ \kappa = \frac{4IL}{{\pi}d_c^2U} \times 10^3 \]

where  \(I\) is the electric current (\si{\ampere}), \(L\) the full length of the capillary (\si{\meter}), \(d_c\) the inner diameter (\si{\meter}) and \(U\) the voltage (\si{\volt}).


\subsection{effective mobility}

The effective mobility (\si{\meter\squared\per\volt\per\second}) reprents the mobility of the compound wihtout taking into consideration the electroosmotic flow. It is calculated with the equation:

\[ \mu_{eff} = \frac{lL}{t_mU} - \mu_{eof} \]

where \(l\) is the length to detection window (\si{\meter}), \(L\) the full length of the capillary (\si{\meter}), \(t_m\) the migration time (\si{\second}), \(U\) the voltage (\si{\volt}) and \(\mu_{eof}\) the electroosmosis mobility (\si{\meter\squared\per\volt\per\second}).


\bibliography{references}

\end{document}
